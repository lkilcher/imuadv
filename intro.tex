
\section{Introduction}

Acoustic Doppler velocimeters (ADVs) have been used to make high-precision measurements of water velocity for over 20 years \cite[]{Kraus++1994, Lohrmann++1995}.  During that time they have been deployed around the world to measure turbulence from a range of platforms, including: stationary structures on ocean- and lake-bottoms, in surface waters from a pole lowered from a ship's bow, and in the deep ocean from autonomous underwater vehicles \cite[e.g.][]{Voulgaris+Trowbridge1998, Zhang++2001, Kim++2000, Goodman++2006, Lorke2007, Geyer++2008, Cartwright++2009}. 

% \cite{Lueck++2002} has a good review of turbulence measurements. 

A relatively small fraction of ADV measurements have been made from moorings \cite[e.g.][]{Fer+Paskyabi2014}. Presumably this is because mooring motion can contaminate ADV measurements, and acoustic Doppler {\it profilers} (ADPs) can be used to measure mid-depth turbulence statistics without a mooring \cite[e.g.][]{Stacey++1999a, Rippeth++2002, Wiles++2006}. Still, ADV measurements have distinct characteristics that can be advantageous: they are capable of higher sample-rates, they have higher signal-to-noise ratios, and they have a much smaller sample-volume (1 centimeter, as opposed to several meters). That is, the ADV is a high precision instrument compared to an ADP, and it is therefore useful to have a method for making moored ADV measurements that account for mooring motion.

Inertial motion sensors (IMUs) have been used in the aerospace and aeronautical industries to quantify the motion of a wide range of systems---including aircraft, rockets, and spacecraft---for several decades, but their cost has come down as their market has grown beyond these niche sectors \cite[]{Bevly2004}. Over the last ten years massive growth in the smart-phone, drone, and `Internet of Things' markets has driven innovation in micro-electrical-mechanical systems (MEMS). One component that has emerged from this sector is the inertial motion unit (IMU), also known as the `Magnetic, angular-rate, gravity' (MARG), or `attitude heading reference system' (AHRS) sensor. These sensors measure three axes of: the earth's magnetic field, angular rotation, and linear acceleration. These signals are then integrated using Kalman filters to estimate the orientation and motion of the sensor \cite[]{Barshan+Whyte1995, Marins++2001, Bachmann++2003}\footnote{Within this literature, `IMU' is generally reserved for a MARG sensor without a magnetometer, but herein we refer to the entire group of sensors that measure motion using accelerometers and angular-rate sensors as `IMUs'.}.

Nortek now offers a version of their Vector ADV with a Microstrain 3DM-GX3-25 IMU sensor \cite[]{vector_manual2005, MicroStrain2012a}. Nortek incorporated this IMU sensor's signals into the Vector data stream so that the motion and orientation signals were tightly synchronized with the ADV's velocity measurements. This tight synchronization provides a data-stream that can be utilized to quantify ADV motion in the earth's inertial reference frame, and remove that motion from the ADV's velocity measurements at each time-step of the its sampling. This work provides a detailed accounting for performing motion correction of these `ADV-IMU' measurements, and presents results of this method using data from a range of mooring configurations that positioned ADV-IMUs at mid-depths in Puget Sound. 

This effort was originally motivated by a need for low-cost, high-precision turbulence measurements for the emerging tidal energy industry \cite[]{Mccaffrey++2015, Alexander+Hamlington2015}. Experience in the wind energy industry has shown that wind turbine lifetime is reduced by atmospheric turbulence, and the same is expected to be true for tidal energy turbines. In wind, meteorological towers are often used to position sonic anemometers at the hub-height of wind turbines for measuring detailed turbulence inflow statistics \cite[]{Hand++2003, Kelley++2005, Mucke++2011, Afgan++2013}. In the ocean, tower-mounted hub-height turbulence measurements have been made, but they are challenging to install and maintain in energetic tidal sites \cite[]{Gunawan++2014}. Thus, the Department of Energy funded this work to investigate the accuracy of mooring deployed ADV-IMUs to reduce the cost of turbulence measurements at tidal energy sites \cite[]{Kilcher++2016}. The approach proved to be successful and potentially useful to the broader oceanographic community interested in moored turbulence measurements \cite[]{Lueck+Huang1999, Doherty++1999, Nash++2004, Moum+Nash2009b, Alford2010, Paskyabi+Fer2013}.

The next section describes details of the measurements, including a summary of the hardware configurations (platforms) that were used to support and position the ADV-IMUs in the water-column. A detailed description of the motion of these platforms is found in the campanion paper to this work, \citeauthor{Harding++2017} (2017), hereafter `Part 1'. Section \ref{sec:methods} describes the mathematical details of motion correction, and section \ref{sec:results} presents results from application of the method to measurements from the various platforms. Section \ref{sec:discussion} is a discussion of the energetics of the tidal channel where the measurements were made and demonstrates that the measurements are consistent with turbulence theory and other measurement in similar regimes. A summary and concluding remarks is found in section \ref{sec:conclusion}.

%%% Local Variables:
%%% mode: latex
%%% TeX-master: "Kilcher_etal_IMU-ADV"
%%% End:

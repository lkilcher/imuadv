
\section{Discussion}

The beginning of the previous section presented a comparison of $\vec{\bar{u}}$ measured by a TTM-mounted ADV, to measurements from a co-located ADP. This demonstrated that the IMU provides a reliable estimate of the ADV's orientation and that this can be utilized to estimate mean velocity in the earth's reference frame. Ideally, moored turbulent velocity measurements would be compared to an independent measurement at the same location.


The noise-level of the ADP measurements was too large to 
A comparison of ADP-measured turbulence revealed that the spectra estimate of those measurements.   fluctuations from would be made. However, because ADP's do not provide estimates of turbulence velocity at the scales 

\begin{figure*}[t]
  \centering
  \includegraphics{TurbTime_TTM_01}
  \caption{Time-series of mean velocities (A), turbulence energy and its components (B), Reynold's stresses (C), and turbulence dissipation rate (D) measured by the TTM during the June, 2014 deployment. Shading indicates periods of ebb ($\bar{u}>1.0$, grey), and flood ($\bar{u}<-1.0$, lighter grey).}
  \label{fig:turbtime:ttm}
\end{figure*}

\begin{figure*}[t]
  \centering
  \includegraphics{EpsVU_TTM_02}
  \caption{$\epsilon$ versus $\bar{U}$ for the June 2014 TTM deployment during ebb (left), and flood (right). Small points are 5 minute averages, and their color indicates the angle of the mean horizontal velocity relative to the principal ebb or flood direction ($\Delta\theta=\theta-\theta_\circ$, where $\theta$ is the horizontal velocity direction, and $\theta_\circ$ is 310$^\circ$ and 130$^\circ$ true for ebb and flood, respectively).  Green dots are mean values within speed bins of 0.2 m s$^{-1}$ width that have at least 6 points (30 minutes of data); their vertical bars are 95\% bootstrap confidence intervals. The black line shows a $U^3$ slope, where the proportionality constant (grey box) is calculated by taking the log-space mean of $\epsilon/U^3$. }
  \label{fig:epsVu:ttm}
\end{figure*}

\begin{figure*}[t]
  \centering
  \includegraphics{EpsVU_SM_02}
  \caption{$\epsilon$ versus $\bar{U}$ for the May 2015 StableMoor deployment during ebb (left), and flood (right). The markers and annotations are identical to figure \ref{fig:epsVu:ttm}. }
  \label{fig:epsVu:sm}
\end{figure*}

This discussion needs to focus on turbulence measurements.

For many applications, such as estimating the turbulence dissipation rate, or turbulent kinetic energy, estimating the spectra is an intermediate step...
\begin{itemize}
\item $\uhead$ provides a justification, and a screening criteria(?), for removing portions of the signal.
\item Can estimate $\epsilon$, tke with fits that exclude contaminated portions of the spectra.
\item Can remove up to 10x signal motion
\item Discuss the time-domain method vs. spectral (cross-coherence) methods, and the need for phase information.
\item Should we estimate $\epsilon$ in this paper? If so, should this be in the Results section?
\end{itemize}



%%% Local Variables:
%%% mode: latex
%%% TeX-master: "Kilcher_etal_IMU-ADV"
%%% End:

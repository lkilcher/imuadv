
\section{Discussion}

\begin{figure*}[t]
  \centering
  \includegraphics{TurbTime_TTM_01}
  \caption{Time-series of mean velocities (A), turbulence energy and its components (B), Reynold's stresses (C), and turbulence dissipation rate (D) measured by the TTM during the June, 2014 deployment. Shading indicates periods of ebb ($\bar{u}>1.0$, grey), and flood ($\bar{u}<-1.0$, lighter grey).}
  \label{fig:turbtime:ttm}
\end{figure*}

The beginning of the previous section presented a comparison of $\vec{\bar{u}}$ measured by a TTM-mounted ADV, to measurements from a co-located ADP. This demonstrated that the IMU provides a reliable estimate of the ADV's orientation and that this can be utilized to estimate mean velocity in the earth's reference frame. Turbulence velocity estimates from the same ADP are also in agreement with low-frequency TTM turbulence estimates (not shown), but the ADP does not resolve turbulence at the scales where motion contamination is strongest (0.1 to 1.0 Hz).

Ideally, moored motion-corrected turbulence velocity measurements would be validated against simultaneous independent validated measurements of turbulence velocity at the same scales and exact location. Accomplishing this, however, involves significant technical challenges not easily overcome---most notable is the challenge of measuring turbulence at the same point as the moving ADV. A slightly less ideal but much more realistic confirmation of the methodology might involve comparing the statistics of moored turbulence measurements to that from a nearby fixed platform, but such measurements have not yet been made \citep[e.g. the `TTT' platform described in][]{Thomson++2012}.

Lacking an independent stationary turbulence measurement near the moored measurements, it is instructive to demonstrate the degree to which the moored measurements are consistent with turbulence theory and other turbulence measurements in similar flow environments. The previous section showed that the shape of the turbulence velocity spectra from moored ADVs is consistent with Kolmogorov's theory of locally isotropic turbulence, which has been observed consistently in turbulence measurements for decades \citep[]{Kolmogorov1941c,Grant++1962,McMillan++2016}. In particular, we observed an isotropic subrange---an $f^{-5/3}$ spectral slope, and equal amplitude spectra between components---that is driven by anisotropic turbulence at longer time-scales (Figures \ref{fig:spec:ttm}, \ref{fig:spec:sm}, \ref{fig:spec:torpedo}). This is interpreted as the first indication that the measurement systems presented are capable of accurately resolving turbulence. The degree to which uncorrected spectra were corrected toward this theoretical and observationally confirmed shape is interpreted as a measure of the improvement of the spectral estimates by motion correction.

Figure \ref{fig:turbtime:ttm} presents a time-series of the mean velocity (A) and several turbulence statistics that have were measured during the June 2014 TTM deployment. This figure shows the evolution of the flow through Admiralty Inlet during 1.5 tidal cycles. The $\tke$ (B), Reynold's stresses (C), dissipation and one component of turbulence production (D) grow and strengthen with ebb or flood, then subside during slack tide.  This component of turbulence production is:
\begin{align}
  \produz = \frac{\partial \bar{u}}{\partial z}\uw \qquad .
\end{align}
Where $\partial \bar{u}/\partial z$ is computed from the $\bar{u}$ measurements from the two ADV's on the TTM, which are separated vertically by 0.5 m. The highest values of $\epsilon$ and $\produz$ occur in the middle of the ebb or flood, which is in agreement with other measurements in tidal channels. The agreement of the magnitude of the $\produz$ with $\epsilon$ at those times suggests a local production-dissipation balance that is often observed in tidally forced channels \citep[]{Trowbridge++1999,Stacey++1999,McMillan++2016}. At other times the value of $\produz$ is insufficient to balance $\epsilon$ or is negative.

\begin{figure}[t]
  \centering
  \includegraphics{EpsVProd01}
  \caption{$\produz$ vs. $\epsilon$ during the June 2014 TTM deployment. Values of `negative' production are indicated as open circles. }
  \label{fig:prodVeps}
\end{figure}

Inspection of the negative $\produz$ values reveals that the vast majority of them are due to a reversed sign of \uw, rather than a reversed sign of $\partial u / \partial z$. This suggests that uncertainty in $\uw$ may be contributing to discrepancies between $\produz$ and $\epsilon$. It is also possible that other terms of the $\tke$ equation are important, such as other components of production, advection terms, or turbulent transport terms.

Figure \ref{fig:prodVeps} compares individual values of $\produz$ with $\epsilon$ directly. Given the assumptions implicit in this comparison, and the discussion above, the agreement between $\produz$ and $\epsilon$ is an encouraging result that suggests the turbulent boundary reaches the depth of these measurements (10 m) during the highest flow speeds. This results is further supported by a comparison of $\bar{U}$ with $\epsilon$ (Figure \ref{fig:epsVu}). Here we see a $\epsilon \propto \bar{U}^3$ dependence that is again suggestive of bottom boundary layer physics \cite{Trowbridge1992,Nash++2009}. At lower flow speeds, $\epsilon$ deviates from this relationship, which suggests that the boundary layer is no longer the dominant physical process at the depth of these measurements.


\begin{figure*}[t]
  \centering
  \includegraphics{EpsVU_03}
  \caption{A log-log plot of $\epsilon$ versus $\bar{U}$ for the June 2014 TTM (diamonds) and May 2015 StableMoor (dots) deployments, during ebb (left) and flood (right). Black points are 5 minute averages.  Green dots are mean values within speed bins of 0.2 m s$^{-1}$ width that have at least 10 points (50 minutes of data); their vertical bars are 95\% bootstrap confidence intervals. The blue line shows a $U^3$ slope, where the proportionality constant (blue box) is calculated by taking the log-space mean of $\epsilon/U^3$. }
  \label{fig:epsVu}
\end{figure*}



%%% Local Variables:
%%% mode: latex
%%% TeX-master: "Kilcher_etal_IMU-ADV"
%%% End:

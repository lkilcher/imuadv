\section{Methods}


This provides an estimate of the low-frequency motion of the platform that is not contaminated by the bias-drift inherent to accelerometer based measurements of motion. 

By applying complimentary and high-pass filters to the ADP's bottom track measurements and $\uacc$ measurements, respectively, the two signals can be combined to provide a complete estimate of the platform's translational motion.


\section{Results}

Furthermore, the shape of $\spec{\ue}$ are generally consistent with similar measurements of turbulence in energetic tidal channels from stationary platforms \citep[]{Thomson++2010} . They have constant amplitude at low frequencies, and roll off to a $f^{-5/3}$ slope at high frequencies that is consistent with Kolmogorov's theory of locally isotropic turbulence. 

Figure \ref{fig:spec01} presents motion-corrected spectra from the Tidal Turbulence Mooring (blue), and compares them with the uncorrected spectra (black) and the spectra of ADV head motion (red) for different mean-flow speeds.  

This motion contaminates the uncorrected $u$ and $v$ spectra for all ranges of the mean velocity, and begins to contaminate the $w$ spectra as the mean velocity increases beyond 2 m/s. 

A notable deviation from this shape is a peak in the $v$-component spectra at 0.1 to 0.2 Hz, especially in the $v$. The $w$-component motion spectrum, $S{w_\mathrm{h}}$ of this platform is relatively unaffected by motion. are relatively few 

While it is encouraging that the moored IMU-ADV is capable of measuring mean velocity, ADVs are typically deployed for the purpose of measuring turbulence--not the mean, which is more easily measured by Doppler profilers. 

Though there are some periods where the $v$ component of velocity differs by as much as 0.4m/s, the $u$ component is always within 0.1m/s.

The mean velocity vector from the TTM-based ADV measurements agrees remarkably well with the velocity measured by the upward looking, bottom-mounted Doppler profiler at the same depth (Figure \ref{fig:vel_time}). 



\section{Introduction}

Acoustic Doppler velocimeters (ADVs) have been used to make high-precision measurements of water velocity for over 20 years \cite[]{Kraus++1994, Lohrmann++1995}.  During that time, they have been deployed around the world to measure turbulence from a range of platforms, including stationary structures on ocean- and lake-bottoms, in surface waters from a pole lowered from a ship's bow, and in the deep ocean from autonomous underwater vehicles \cite[e.g.,][]{Voulgaris+Trowbridge1998, Zhang++2001, Kim++2000, Goodman++2006, Lorke2007, Geyer++2008, Cartwright++2009}. 

% \cite{Lueck++2002} has a good review of turbulence measurements. 

A relatively small fraction of ADV measurements have been made from moorings \cite[e.g.,][]{Fer+Paskyabi2014}. Presumably this is because mooring motion can contaminate ADV measurements, and acoustic Doppler profilers (ADPs) can be used to measure mid-depth turbulence statistics without a mooring \cite[e.g.,][]{Stacey++1999a, Rippeth++2002, Wiles++2006}. Still, ADV measurements have distinct characteristics that can be advantageous: they are capable of higher sample rates, have higher signal-to-noise ratios, and have a much smaller sample volume (1 centimeter, as opposed to several meters). That is, compared to an ADP, ADVs are high-precision instruments capable of providing unique information. They could be more widely used as a moored instrument (i.e., at an arbitrary depth) if a method for accounting for mooring motion can be demonstrated to provide more accurate estimates of turbulence statistics.

Inertial motion unit (IMU) sensors have been used in the aerospace and aeronautical industries to quantify the motion of a wide range of systems for several decades \cite[]{Bevly2004}. Over the last 10 years, the smartphone, drone, and `Internet of Things' markets has driven innovation in microelectrical-mechanical systems, including the IMU. As a result of this growth and innovation the cost, power requirements, and size of IMUs have come down. Also known as MARG (magnetic, angular-rate, gravity), or AHRS (attitude heading reference system) sensors, IMUs measure three axes of the Earth's magnetic field, angular rotation, and linear acceleration.\footnote{Within this literature, IMU is generally reserved for a MARG sensor without a magnetometer, but herein we refer to the entire group of sensors that measure motion using accelerometers and angular-rate sensors as IMUs.} These signals are then integrated using Kalman filters to estimate the orientation and motion of the sensor \cite[]{Barshan+Whyte1995, Marins++2001, Bachmann++2003}.

Nortek now offers a version of their Vector ADV with a Microstrain 3DM-GX3-25 IMU sensor \cite[]{vector_manual2005, MicroStrain2012a}. The IMU's signals are incorporated into the Vector data stream so that the motion and orientation signals are tightly synchronized with the ADV's velocity measurements. This tight synchronization provides a data stream that can be utilized to quantify ADV motion in the Earth's inertial reference frame, and remove that motion from the ADV's velocity measurements at each time step of its sampling. This work specifies a method for performing motion correction of these `ADV-IMU' measurements, and presents results of this method using data from a range of mooring configurations that positioned ADV-IMUs at mid-depths in Puget Sound. 

This effort was originally motivated by a need for low-cost, high-precision turbulence measurements for the emerging tidal energy industry \cite[]{Mccaffrey++2015, Alexander+Hamlington2015}. Experience in the wind energy industry has shown that wind turbine lifetime is reduced by atmospheric turbulence, and the same is expected to be true for tidal energy turbines. In wind, meteorological towers are often used to position sonic anemometers at the hub height of wind turbines for measuring detailed turbulence inflow statistics \cite[]{Hand++2003, Kelley++2005, Mucke++2011, Afgan++2013}. In the ocean, tower-mounted hub-height turbulence measurements have been made, but they are challenging to install and maintain in energetic tidal sites \cite[]{Gunawan++2014,Thomson++2012}. Therefore, the U.S. Department of Energy funded this work to investigate the accuracy of mooring-deployed ADV-IMUs to reduce the cost of turbulence measurements at tidal energy sites \cite[]{Kilcher++2016}. The approach proved to be successful and potentially useful to the broader oceanographic community interested in moored turbulence measurements \cite[]{Lueck+Huang1999, Doherty++1999, Nash++2004, Moum+Nash2009b, Alford2010, Paskyabi+Fer2013}.

The next section describes details of the measurements, including a summary of the hardware configurations (platforms) that were used to support and position the ADV-IMUs in the water column. A detailed description of the motion of these platforms is found in the companion paper to this work, \citeauthor{Harding++2017} (in review), hereafter Part 1. Section \ref{sec:methods} describes the mathematical details of motion correction and Section \ref{sec:results} presents results from applying the method to measurements from the various platforms. Section \ref{sec:discussion} is a discussion of the energetics of the tidal channel in which the measurements were made and demonstrates that the measurements are consistent with turbulence theory and other measurements in similar regimes. A summary and concluding remarks are provided in Section \ref{sec:conclusion}.


%%% Local Variables:
%%% mode: latex
%%% TeX-master: "Kilcher_etal_IMU-ADV"
%%% End:

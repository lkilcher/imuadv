
\section{Introduction}

Acoustic Doppler velocimeters (ADVs) have been used to make high-precision measurements of water velocity for over 20 years \cite[]{Kraus++1994, Lohrmann++1995}.  During that time, they have been deployed around the world to measure turbulence in a range of environments and from a range of platforms, including the laboratory setting \cite[]{Voulgaris+Trowbridge1998}, from stationary structures on ocean-, river- and lake-bottoms \cite[]{Kim++2000, Lorke2007, Cartwright++2009}, in surface waters from a pole lowered from a ship's bow \cite[]{Geyer++2008}, and in the deep ocean from autonomous underwater vehicles \cite[e.g.,][]{Zhang++2001, Goodman++2006}. 

A relatively small fraction of ADV measurements have been made from moorings \cite[e.g.,][]{Fer+Paskyabi2014}. Presumably this is because mooring motion can contaminate ADV measurements, and acoustic Doppler profilers (ADPs) can measure some mid-depth turbulence statistics without a mooring \cite[e.g.,][]{Stacey++1999a, Rippeth++2002, Wiles++2006, Guerra+Thomson2017}. Still, ADV measurements have distinct advantages: they are capable of higher sample rates, have higher signal-to-noise ratios, and have a much smaller sample volume (1 centimeter, as opposed to several meters). 

Inertial motion unit (IMU) sensors have been used in the aerospace and aeronautical industries to quantify the motion of a wide range of systems, and to improve atmospheric velocity measurements, for several decades \cite[]{Axford1968, Edson++1998, Bevly2004}. In the last decade, the smartphone, drone, and `Internet of Things' markets have driven innovation in microelectrical-mechanical systems, including the IMU. As a result of this growth and innovation, the cost, power requirements, and size of IMUs have come down. These changes have allowed these sensors to be integrated into oceanographic instruments that have small form-factors, and rely on battery power.

Nortek now offers a version of their Vector ADV with a Microstrain 3DM-GX3-25 IMU sensor \cite[]{vector_manual2005, MicroStrain2012a}. This IMU's signals are incorporated into the Vector data stream so that its motion and orientation signals are tightly synchronized with the ADV's velocity measurements. The tight synchronization provides a dataset that can be utilized to quantify ADV motion in the Earth's inertial reference frame, and remove that motion from the ADV's velocity measurements at each time step of its sampling \cite[]{Edson++1998}. This work utilizes `ADV-IMU' measurements from mid-depth moorings in Puget Sound to demonstrate that motion correction can improve the accuracy of oceanic turbulence spectra, turbulence dissipation, and Reynolds stress estimates.

This effort was originally motivated by a need for low-cost, high-precision turbulence measurements for the emerging tidal energy industry \cite[]{Mccaffrey++2015, Alexander+Hamlington2015}. Experience in the wind energy industry has shown that wind turbine lifetime is related to atmospheric turbulence, and the same is expected to be true for tidal energy turbines. In the atmosphere, meteorological towers are often used to position sonic anemometers at the hub height of wind turbines for measuring detailed turbulence inflow statistics \cite[]{Hand++2003, Kelley++2005, Mucke++2011, Afgan++2013}. In the ocean, tower-mounted hub-height turbulence measurements have been made, but they are challenging to install and maintain in energetic tidal sites \cite[]{Gunawan++2014,Thomson++2012}. Therefore, the U.S. Department of Energy funded this work to investigate the accuracy of mooring-deployed ADV-IMUs to reduce the cost of turbulence measurements at tidal energy sites \cite[]{Kilcher++2016}. The approach proved to be successful and potentially useful to the broader oceanographic community interested in moored turbulence measurements \cite[]{Lueck+Huang1999, Doherty++1999, Nash++2004, Perlin+Moum2012, Alford2010, Paskyabi+Fer2013}.

The next section describes details of the measurements, including a summary of the hardware configurations (platforms) that were used to support and position the ADV-IMUs in the water column. A detailed description of the motion of these platforms is found in the companion paper to this work, \cite{Harding++2017}, hereafter Part 1. Section \ref{sec:methods} describes the mathematical details of motion correction and Section \ref{sec:results} presents results from applying the method to measurements from the various platforms. Section \ref{sec:discussion} is a discussion of the energetics of the tidal channel in which the measurements were made and demonstrates that the measurements are consistent with turbulence theory and other measurements in similar regimes. A summary and concluding remarks are provided in Section \ref{sec:conclusion}.


%%% Local Variables:
%%% mode: latex
%%% TeX-master: "Kilcher_etal_IMU-ADV"
%%% End:

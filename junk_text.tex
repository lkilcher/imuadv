\section{Intro}

Since their invention more than twenty years ago acoustic Doppler velocimeters (ADVs) have been used to measure turbulent velocities in a wide range of environments. Originally designed to make point-measurements in laboratory flumes and tanks, ADVs have become an increasingly useful tool in the oceanic environment. They have been deployed, for example, on the seafloor to measure fluxes in the oceanic bottom boundary layer, in the surf-zone to measure Reynold's stresses, and from ships on downward protruding masts to measure surface mixing \citep[e.g.]{Lohrmann++1994, Voulgaris+Trowbridge1998, Kim++2000, Trowbridge+Elgar2003, Elgar++2005, Geyer++2008}.

Why is the turbulence spectrum important? -- Captures motion at all scales.
ADCPs 
1) History of ADV use. How are they used? In what turbulence levels? What is their noise-floor? 
1a) How do they compare to other turbulence tools?
Compared to profilers, ADVs are:
  - Lower precision than profilers at high f?
  - Capture low-frequency information.
  - can be fixed in space.
Compared to ADCP, ADVs can capture the turbulence spectrum:

  - Higher precision at high f
  - Higher resolution (down to cm scales)
  - Lower noise

Say something about: we don't have independent measurements of turbulence to validate against, so we use spectra--and our knowledge of their theoretical properties--to gain confidence in the method.

A primary limitation of ADVs is that it has been challenging to place them at mid-depths (i.e. away from the seafloor and the surface). In the oceanic environment they have either been deployed on fixed platforms anchored to the sea-floor, or deployed near the surface from a ship.  Until now, there has been no method for deploying these instruments at mid-water depths, or near the surface without a ship.  This work describes a new approach for measuring turbulence spectra from ADVs mounted on moorings. Until recently this approach has been limited by the fact that mooring motion contaminates the velocity measurements.  The frequency and amplitude of this contamination will depend on the characteristics of the mooring and of the turbulent environment in which it is deployed. In most cases mooring motion will--over some range of frequencies--be similar to or greater than the turbulence velocities one would like to measure. This `motion contamination' reduces usefulness of the ADV measurements. 

Over the last decade, the cost of inertial motion sensors (IMUs)

However, this being said,  In many cases the motion of the mooring Depending Because When the mooring (and ADV head) moves with the flow the measured velocity will be reduced, and when it moves against the flow it will enhance the it. is often equal to the turbulent velocities one would like to measure. contaminates the velocity measurements  equipped with 

In energetic environments ADVs provide 

In energetic environments, where turbulence levels ADVs provide are Throughout 

Velocity measurements have been made from 

As flexible and useful as these instruments are, one limitation 
% boundary-layer studies. 

are useful tools for making point-measurements of water velocity in labarotory and field environments. 
robust and versatile tools 

  ADV velocity measurements from compliant moorings will be contaminated by mooring motion. When an inertial motion unit (IMU, also known as a magnetic, angular rate, gravity or MARG sensor) is rigidly attached to, and tightly synchronized with, acoustic Doppler velocimeter (ADV) measurements the IMU's orientation and motion measurements can be used to reduce motion contamination.  Microstrain IMU on-board the Nortek Vector is equipped with a 3-axis accelerometer, a 3-axis rotation-rate sensor, and a 3-axis magnetometer.  The Microstrain samples these signals at 1000Hz and performs integration and Kalman filtering operations to produce stable estimates of it's orientation, change in velocity, and change in rotation-rate.  and  can be configured to output estimates of,


\section{Methods}


This provides an estimate of the low-frequency motion of the platform that is not contaminated by the bias-drift inherent to accelerometer based measurements of motion. 

By applying complimentary and high-pass filters to the ADP's bottom track measurements and $\uacc$ measurements, respectively, the two signals can be combined to provide a complete estimate of the platform's translational motion.


\section{Results}

Furthermore, the shape of $\spec{\ue}$ are generally consistent with similar measurements of turbulence in energetic tidal channels from stationary platforms \citep[]{Thomson++2010} . They have constant amplitude at low frequencies, and roll off to a $f^{-5/3}$ slope at high frequencies that is consistent with Kolmogorov's theory of locally isotropic turbulence. 

Figure \ref{fig:spec01} presents motion-corrected spectra from the Tidal Turbulence Mooring (blue), and compares them with the uncorrected spectra (black) and the spectra of ADV head motion (red) for different mean-flow speeds.  

This motion contaminates the uncorrected $u$ and $v$ spectra for all ranges of the mean velocity, and begins to contaminate the $w$ spectra as the mean velocity increases beyond 2 m/s. 

A notable deviation from this shape is a peak in the $v$-component spectra at 0.1 to 0.2 Hz, especially in the $v$. The $w$-component motion spectrum, $S{w_\mathrm{h}}$ of this platform is relatively unaffected by motion. are relatively few 

While it is encouraging that the moored IMU-ADV is capable of measuring mean velocity, ADVs are typically deployed for the purpose of measuring turbulence--not the mean, which is more easily measured by Doppler profilers. 

Though there are some periods where the $v$ component of velocity differs by as much as 0.4m/s, the $u$ component is always within 0.1m/s.

The mean velocity vector from the TTM-based ADV measurements agrees remarkably well with the velocity measured by the upward looking, bottom-mounted Doppler profiler at the same depth (Figure \ref{fig:vel_time}). 


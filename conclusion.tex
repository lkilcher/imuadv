
\section{Conclusion}
\label{sec:conclusion}
 
This work presents a methodology for measuring turbulence from moored ADV-IMUs and demonstrates that motion correction reduces mooring motion-contamination. Comparison of spectra of ADV head motion, $\spec{\uhead}$, to that of motion-corrected, $\spec{\ue}$, and uncorrected spectra, $\spec{\umeas}$, reveals that motion correction improves spectral estimates of moored ADV measurements. In particular, we found that motion-corrected spectra have spectral shapes that are similar to previous measurements of tidal-channel turbulence and have a $f^{-5/3}$ spectral slope at high frequencies. This finding suggests that the motion-corrected spectra resolve the inertial subrange predicted by Kolmogorov's theory of locally isotropic turbulence.

Motion correction reduces motion contamination for all platforms we presented but it does not necessarily remove it completely. This outcome seems to depend on the relative amplitude of platform motion compared to the underlying turbulence being measured. The most notable example of this is from the TTM $v$-component spectra, which have a large-amplitude ``swaying'' peak at 0.15 Hz that interrupts the often observed `roll-off' between the low-frequency `energy containing scales' and the $f^{-5/3}$ inertial subrange.

This inconsistency indicates that turbulence measurements from moored, motion-corrected ADV-IMUs must be interpreted with care. An inspection of spectra presented here suggests that excluding spectral regions where $\spec{\uhead} / \spec{\ue} > 3$ removes persistent-motion contamination peaks while still preserving spectral regions where motion correction is effective. Using this criteria, it is then possible to produce spectral fits that exclude persistent-motion contamination, and provide reliable estimates of turbulence quantities of interest (e.g., $\epsilon$ and $\tke$).

We have also shown that motion correction reduces motion contamination in cross spectra. This finding is important because it suggests that moored ADV-IMU measurements may be used to produce reliable estimates of Reynolds stresses. We utilized these stress estimates and vertical shear estimates, both from the TTM, to estimate $\produz$. 

Finally, we have shown that $\epsilon$ estimates based on motion-corrected spectra scale with the $U^3$, and balance $P_{uz}$ estimates during peak ebb and flood. Together, these results indicate that bottom boundary layer physics are a dominant process at this site, and that the boundary layer reaches the height of the ADV-IMUs (10 m) during ebb and flood. The degree of agreement between $\produz$ and $\epsilon$ also serves as an indicator of the self-consistency of moored ADV-IMU turbulence measurements. 

%%% Local Variables:
%%% mode: latex
%%% TeX-master: "Kilcher_etal_IMU-ADV"
%%% End:
